\documentclass{article}

\begin{document}
\section{Ray-Sphere Intersection}

\paragraph{}
Equation of a Line: \(x = o + d l \)

\begin{itemize}
\item \(x\): a point along the line
\item \(o\): the origin point of the line
\item \(d\): the distance along the line
\item \(l\): the direction vector of the line (normalized)
\end{itemize}

\paragraph{}
Equation of a Sphere: \(\|x - c\|^2 =r^2\)

\begin{itemize}
\item \(x\): a point on the sphere
\item \(c\): the center point on the sphere
\item \(r\): the radius of the sphere
\end{itemize}


\paragraph{}
Math Occurs

\[d = -(l \cdot (o - c)) \pm \sqrt{(l \cdot (o - c))^2 - (\|o - c\|^2 - r^2)}\]

Notes:

\begin{itemize}
\item For rays, all intersections must be \(d \geq 0\) since anything outside of
  that is outside of the ray.
\item If the term inside the square root is negative, the line does not
  intersect the sphere.
\item If the term is zero, there is exactly a single intersection. Otherwise,
  there are two.
\end{itemize}

\section{Ray-Plane Intersection}

Equation of a Line: \(x = o + d l \)

\begin{itemize}
\item \(x\): a point along the line
\item \(o\): the origin point of the line
\item \(d\): the distance along the line
\item \(l\): the direction vector of the line (normalized)
\end{itemize}

\paragraph{}
Equation of a Plane: \((p - p_0) \cdot n = 0\)

\begin{itemize}
\item \(p\): a point on the plane
\item \(p_0\): the origin point of the plane
\item \(n\): the surface normal of the plane
\end{itemize}

\paragraph{}

Math Occurs

\[d = \frac{(p_0 - o) \cdot n}{l \cdot n}\]

Notes:

\begin{itemize}
\item For rays, all intersections must be \(d \geq 0\) since anything outside of
  that is outside of the ray.
\item If \(l \cdot n = 0\), the line is parallel to the plane.
\end{itemize}

\section{Lambertian Light}

\[I_d = l \cdot n C I_l\]

\begin{itemize}
\item \(I_d\): the intensity of diffuse light
\item \(l\): the vector toward the light from the point of intersection
  (normalized)
\item \(n\): the surface normal at the point of intersection
\item \(C\): the color of the surface
\item \(I_l\): the intensity of the light
\end{itemize}

\section{Whitted Model}

\[I = I_a + k_d\displaystyle\sum_{j=0}^{lights} (n \cdot l_j) + k_s S + k_t T\]

\begin{itemize}
\item \(I\): the intensity of light in the Whitted model
\item \(I_a\): the intensity of ambient light (constant in the scene)
\item \(k_d\): the diffuse coefficient
\item \(n\): the surface normal at the point of intersection
\item \(l_j\): the vector from the point of intersection to the light
  (normalized)
\item \(k_s\): the specular coefficient
\item \(S\): the intensity of reflected light 
\item \(k_t\): the transmission coefficient
\item \(T\): the intensity of refracted light
\end{itemize}

\section{Reflection}

\paragraph{}
Equation for a vector that is the component of the surface normal that is
perpendicular to the normal direction: \(perp_nl = l - (n \cdot l)n\)

\[R = 2(n \cdot l)n - l\]

\paragraph{}
This is much easier to explain via a diagram. Mathematics for 3D Game
Programming \& Computer Graphics contains an easy-to-follow diagram and
explanation on page 152-153.

\end{document}